\documentclass[11pt, oneside]{article}
\title{{\tt frag\_asm}: User's guide}
\author{Martin Kochan ({\tt mkochan@ksu.edu})}
\date{25 November, 2005}

\begin{document}
\maketitle

\section{Description}

This program performs simple DNA fragment assembly. 
It was written as part of the ``Implementation Project'' course taken by the author
at Kansas State University, Manhattan, Kansas, in Spring 2005.
The current version is 1.00.

\section{Motivation}

DNA that may be several thousand bases (characters) long. It is impossible to read the whole
string at one time with current technology. However, a method called \emph{shotgun sequencing}
is capable of reading strings of limited length (up to several hundred bases) that are taken at
random places in the original string. These \emph{reads} often contain reading errors, such as
point mutations (changes of characters), additions (when a meaningless character is added), and
deletions (when a meaningful character is deleted).

From these reads, we want to assemble longest possible contiguous segments of the original long
string, called \emph{contigs}. We must account for reading errors!

\section{Synopsis}

The command line looks like:
\begin{verbatim}
./frag_asm seqs.dat seqs.out seqs.progress 
\end{verbatim}

The contents of file {\tt seqs.dat} (input sequences, one per line, separated
by newlines or whitespaces) are:
\small
\begin{verbatim}
cagctagctactgcatcgatgctaccgatcgtaagcccacaccacac        
acgtaggcgctagggctatgctaggctgcggtacgatgccctcgatcgtaagcccacacc
ggtacgatgctcgatcgtaagcccggtgctagctagcatcgatgctagctagct
cccccctattcgatttttgggggggacatt
gctacgatcgatgctagtgctgtaccccccccctattcgatttttggggacaatttttggcccaa
gctagc
\end{verbatim} 
\normalsize

The output file produced {\tt seqs.dat} then is:
\scriptsize
\begin{verbatim}
THERE ARE 6 READS.

Contig #0:
acgtaggcgctagggctatgctaggctgcggta-cgatgccctcgatcgtaagcccacacc (1) 
                 cagcta-gctactgcatcgatgctaccgatcgtaagcccacaccacac (0) 
                             ggta-cgatg--ctcgatcgtaagcccggtgctagctagcatcgatgctagctagct (2) 

Contig #3:
                           cccccctattcgatttttgggggggacatt (3) 
gctacgatcgatgctagtgctgtaccccccccctattcgatttttggggacaatttttggcccaa (4) 

Contig #5:
gctagc (5) 
\end{verbatim}
\normalsize

The progress of the assembly is recorded along with the output, and is found
in the file {\tt seqs.progress}. Note that the {\tt +++++} sign means
merging of contigs and {\tt ====>} means ``the resulting contig follows.''
Note that resulting contigs always take name of the first ``operand'' in the merging operation.
At the beginning, all reads had their own contigs --- or \emph{singlets}.  

\scriptsize
\begin{verbatim}
Contig #3:
cccccctattcgatttttgggggggacatt (3) 
+++++ 
Contig #4:
gctacgatcgatgctagtgctgtaccccccccctattcgatttttggggacaatttttggcccaa (4) 
====> 
Contig #3:
                           cccccctattcgatttttgggggggacatt (3) 
gctacgatcgatgctagtgctgtaccccccccctattcgatttttggggacaatttttggcccaa (4) 
----------------

Contig #0:
cagctagctactgcatcgatgctaccgatcgtaagcccacaccacac (0) 
+++++ 
Contig #1:
acgtaggcgctagggctatgctaggctgcggtacgatgccctcgatcgtaagcccacacc (1) 
====> 
Contig #0:
                 cagcta-gctactgcatcgatgctaccgatcgtaagcccacaccacac (0) 
acgtaggcgctagggctatgctaggctgcggta-cgatgccctcgatcgtaagcccacacc (1) 
----------------

Contig #0:
acgtaggcgctagggctatgctaggctgcggta-cgatgccctcgatcgtaagcccacacc (1) 
                 cagcta-gctactgcatcgatgctaccgatcgtaagcccacaccacac (0) 
+++++ 
Contig #2:
ggtacgatgctcgatcgtaagcccggtgctagctagcatcgatgctagctagct (2) 
====> 
Contig #0:
acgtaggcgctagggctatgctaggctgcggta-cgatgccctcgatcgtaagcccacacc (1) 
                 cagcta-gctactgcatcgatgctaccgatcgtaagcccacaccacac (0) 
                             ggta-cgatg--ctcgatcgtaagcccggtgctagctagcatcgatgctagctagct (2) 
----------------
\end{verbatim}
\normalsize

\section{Notes}

The program solves the above-mentioned problem only partly (with many simplifications
adopted) and certainly not optimally. To make account of the most significant facts:
\begin{itemize}
\item
  Heuristical preprocessing of reads is performed, based on \emph{14-mers}.
  It makes linear-time guess about which reads to put into common contig.  
\item
  Alignment of reads is done using gapped Smith--Waterman. Currently there is no support for banded
  Smith--Waterman, although generation of bands' positions is already implemented internally.  
  To continue along these lines would make for a good optimization. 
\item
  Input must be simple DNA strings delimited by newlines or spaces. Inputs shorter than 14 bases
  are guaranteed to remain singlets. No other chars besides {\tt AGCT} or {\tt agct} are allowed.
\item
  Output is given in form of layouts rather than consensus sequences, for didactic reasons.
\item
  Currently, complementary strands are not supported.
\end{itemize}

\end{document}
